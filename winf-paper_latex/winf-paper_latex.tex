\documentclass[a4j,8pt,twocolumn]{extarticle}

% ---------------

\usepackage{winf-paper}
\usepackage{amsmath}
\usepackage{amssymb}
\usepackage{ascmac}
\usepackage{latexsym}
\usepackage{ulem}
\usepackage[dvipdfmx]{graphicx}

% ---------------

%% 和文題目
\title{情報学ワークショップ\\論文フォーマット}

%% 和文著者
\author{情報 太郎 \qquad 情報 花子}

%% 和文所属
\affiliation{情報大学情報学部}





% ===== 所属が複数の場合 =====
%\author{情報 太郎\DAG{1} \qquad 情報 花子\DAG{1} \qquad 情報 次郎\DAG{2}}
%\affiliation{\DAG{1}情報大学情報学部 \qquad \DAG{2}情報大学大学院情報学研究科}
%\eauthor{Taro Info\DAG{1} \qquad Hanako Info\DAG{1} \qquad Jiro Info\DAG{2}}
%\eaffiliation{%
%	\DAG{1} Faculty of Information, Information University\\
%	\DAG{2} Graduate School of Information, Information University
%}

\begin{document}
	
\maketitle
\thispagestyle{empty}	% 1ページ目のページ番号を表示しない
% ------------------------------------------------------------

\section{はじめに}

情報学ワークショップは,東海地区を中心として大学・企業等の情報技術に関する最新の研究成果を発表すると共に,学生・研究者の交流の場を提供することを目的としております.本年度は研究論文発表及びポスター発表で行います.研究論文の応募に際しては,2段組4ページ以内とし,ポスター発表に関しては300字の抄録のみでの申込となります.研究論文に提出いただいた論文は,例年通り審査委員により評価します.さらに提出論文の中から特に優秀な論文を選考しこれを表彰します.情報学分野の情報交換の場として多くの方のご参加を期待しております.

本資料では情報学ワークショップの研究論文の作成方法を示します.
発表者の方は,この資料に準拠して研究論文の原稿を作成していただくようお願いいたします.
ただし,それぞれの専門分野で優先すべきフォーマットなどがありましたら,そちらのフォーマットに従うマイナーチェンジも結構です.


% ------------------------------------------------------------
\section{提出いただくもの}
\subsection{研究論文の場合}
抄録(300 文字)と研究論文(A4・4ページ以内)を提出して頂きます.

\subsection{ポスター発表の場合}
抄録(300 文字)のみを申込時に提出して頂きます.

% ------------------------------------------------------------
\section{論文本体のフォーマット概要}

基本的にこのフォーマットに準拠していただきますが,もちろん書きやすいように,読みやすいようにマイナーチェンジして頂いてもOKです.
ただし最終的に書式を実行委員会・プログラム委員会で統一させて頂く場合があります.
なお,この \LaTeX 版のテンプレートは,MS-Word版と異なる点もあります.
以下の各項目を目安として考えていただければと思います.

\begin{enumerate}
\item 日本語タイトル,著者名,所属,最初に1段組みで書きます.
文字サイズは,日本語タイトル12ポイント,日本語の著者名と所属10.5ポイントです.
この間の行間は15ポイントの固定値になっています.

\item 本文は2段組みで,フォントは8ポイントで行間は1行です.
この結果,1ページがおおよそ28文字×57行x2段となります.

\item 各章の見出しは10ポイント,各節の見出しは9ポイントとしています.

\item マージンは上下が3cm,左右が2cmとします.

\item MS-Word版では,文字フォントは,タイトル,著者名,所属,見出しの部分が,MSゴシックとArialを使用しています.
また本文は,MS明朝とCenturyです.
ただしフォントについては特に制限はいたしませんので,独自の形式で論文を作成して構いません.

\item \underline{ページ番号をつけないでください.}

\item この資料は,\LaTeX の範囲内で,おおむねフォーマットに従っているつもりです.
\end{enumerate}


% ------------------------------------------------------------
\section{提出締め切り}

WiNFウェブサイトにあります期限までに,フォーマットに準拠した研究論文のPDFファイルをWiNFウェブサイトの「発表申込」ページから投稿してください.
PDFファイル名に特に指定はございません.


% ------------------------------------------------------------
\begin{thebibliography}{9}
\bibitem{KMST:2011}
河辺 義信, 真野 健, 櫻田 英樹, 塚田 恭章:
電子投票プロトコルに対する無証拠性の定理証明,
情報処理学会論文誌,
Vol. 52, No. 9, pp. 2549-2561 (2011).

\bibitem{HKS:2010}
Ichiro Hasuo, Yoshinobu Kawabe, Hideki Sakurada:
Probabilistic anonymity via coalgebraic simulations,
Theoretical Computer Science,
Vol. 411, No. 22-24, pp. 2239-2259 (2010).
\end{thebibliography}


% ------------------------------------------------------------
\end{document}
