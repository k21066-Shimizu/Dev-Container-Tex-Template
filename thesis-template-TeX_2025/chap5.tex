\chapter{おわりに}
\thispagestyle{myheadings}

\section{まとめ}
本研究では合成音声ライブラリが多く存在する中で,ユーザが用途に合った声を探索するための手法として,声の印象を数値化し,それを用いてライブラリを探索するWebサービス「声色見本帳」を提案した.
まず,UTAU音源の声から声の印象を数値化する機械学習モデルを構築した.
モデルは入力データとして音声から抽出したMFCCをはじめとする音響特徴量を,出力データとして声の印象を7つの評価軸ごとに数値化した評価スコアを取る.
モデル構築にはPythonライブラリであるPyCaretを用い,学習方法にはAdaBoost Regressorを,教師データとして声に対する既存のアンケート結果から168ライブラリ分のデータを用いた.
構築したモデルの推論精度を確認するため評価実験を行い,評価軸ごとの相関係数が概ね0.5〜0.6となる結果を得,一定の精度での推定を確認した.
一方で,「声の年齢」をはじめとする複数の評価軸では予測値が評価スコア範囲中央に偏るなどの問題が見られた.

ユーザが求める声の印象を評価スコアとして入力し,声の近いライブラリを探索できるWebサービス「声色見本帳」を実装した.
ライブラリ探索に用いるライブラリごとの評価スコアは,先立って構築した機械学習モデルを用いて事前に推定し付与する.
サービスはユーザの求める声に近いライブラリを複数提案し,それらのサンプル音声やダウンロードページへのリンクを提供する.
気軽に利用できるWebサービスを通じて目的に適した声との出会いを促進し,求める声質を持つライブラリを効率的に探索できたり,埋もれがちな多様なライブラリへ活用機会の増加が期待できる.

\section{今後の課題}
今後の課題としては,まず機械学習モデルの改善が挙げられる.
改善手法として,より多角的な音響特徴量の利用が考えられる.
本研究でモデルの入力として用いた音響特徴量は,主に母音の発声から特徴量を抽出している.
しかし,子音にまつわる特徴量や,子音を含む音素間の遷移にまつわる特徴量などから声の印象を得られると考えられる.
これらを用いてモデルの入力データの表現力を向上させれば,より適切に声の印象を推定できる.

また,運用されるサービスを活用し,学習データの拡充も考えられる.
例えば,ユーザの探索履歴を収集し,あるユーザの探索パラメータと実際にダウンロードページにアクセスした音源との関連付けを行い,学習データとしての利用が考えられる.
あるいはより直接的に,音源ごとにユーザから評価の投票を募り,その結果を用いるなど,ユーザからのフィードバックを利用した手法も考えられる.
学習データをより多様で正確なものにすれば,モデルの汎用性向上が期待できる.

サービスとしての利便性向上も課題である.
現在,ライブラリの新規追加時にはUTAU音源ラファイルをzipファイルでアップロードする形式であるが,このファイルは数百MBのものも多く,アップロードに時間がかかるなどの問題がある.
サービスの運用にはこの音源ファイルそのものは必要ないため,ユーザのローカル環境上で音声ファイルの解析とサンプル音声の生成が出来れば,その結果を送るのみで済むため通信量の大幅な削減ができる.
探索システムの改善としては,現在評価スコアの類似度指標としてユークリッド距離を用いているが,それについて評価実験と検討が必要と感じている.
より本当にユークリッド距離は評価スコアの類似度を適切に表現できるのか,また他の指標を用いた場合の探索結果の変化について検討が必要である.
また,ライブラリ情報の更新・修正機能,あるいは権利者からの削除依頼への適切な対応体制の整備も必須である.これらの改善を行いより利便性の高いサービスを目指す.

% Local Variables:
% mode: japanese-LaTeX
% TeX-master: "root"
% End:
