\chapter{おわりに}
\thispagestyle{myheadings}

\section{まとめ}
% 本研究では,UTAU音源ライブラリの声質を複数の印象軸で評価スコア化し,それを用いてユーザの求める声に近いライブラリを探索できるサービスを提案した.
% サービスの構築に当たり,アンケートによって得られた評価スコアを目標とし,音声から特徴量を抽出してスコアを推定する機械学習モデルを構築した.
本研究では,まずアンケートによって得られた評価スコアを目標とし,音声から特徴量を抽出してスコアを推定する機械学習モデルを構築した.
その結果,声の強さなど一部の評価軸については相関係数0.66と一定の精度で推定できたものの,声の年齢など他の評価軸では予測値が中央に偏るなどの問題が見られた.
構築したモデルを用いて,ユーザが声の印象をスコアとして入力し,それに近いライブラリを探索できるWebサービス「声色見本帳」を実装した.
気軽に利用できるWebサービスを通じて目的に適した声との出会いを促進し,求める声質を持つライブラリを効率的に探索できたり,埋もれがちな多様なライブラリへ活用機会の増加が期待できる.

\section{今後の課題}
今後の課題としては,機械学習モデルの改善が挙げられる.
\ref{sec:eval}節で述べたように音響特徴量と学習手法の選定による改善のほか,運用されるサービスの活用も考えられる.
例えば,ユーザの探索履歴を収集し,あるユーザの探索パラメータを実際にダウンロードページにアクセスされたUTAU音源の評価に影響させたり,直接ユーザに評価の投票を促しその結果を用いて追加で学習を行うなど,ユーザからのフィードバックを利用した推定精度の向上手法も考えられる.
サービスとしての利便性向上も課題である.
探索時に用いるライブラリ間の評価スコアの類似度指標の改善や,ライブラリ情報の更新・修正機能の実装,また権利者からの削除依頼への適切な対応体制の整備などが考えられ,これらの改善を行うことでより利便性の高いサービスを目指す.

% Local Variables:
% mode: japanese-LaTeX
% TeX-master: "root"
% End:
