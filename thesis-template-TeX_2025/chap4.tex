\chapter{声色見本帳}
\thispagestyle{myheadings}

\section{サービスの概要}
本研究で提案するサービス「声色見本帳」は,ユーザが求める声の評価スコアを入力し,その評価スコアに近い合成音声ライブラリを探索・提案するサービスである.
ユーザは自身が求める声に近いライブラリを探すことができ,多くのライブラリの声を手探りで聞き比べる手間なく自身に合った声を見つけられる.
ライブラリの作成者も,自身のライブラリを多くのユーザに知ってもらう機会を得られる.
サービスはWebサービスとして実装し,ユーザはブラウザから手軽に利用できる.

\subsection{要求仕様}
本研究で提案するサービスについての要求仕様を以下に示す.
ユーザは求める声の評価スコアを入力し,その評価スコアに近いUTAU音源ライブラリを探索できる.
探索結果として提示されたライブラリのサンプル音声を再生したり,配布ページへのリンクを確認できる.
また,ユーザが新しいライブラリを追加できる機能も提供する.
この機能にり,サービスに登録されていなかったライブラリや,今後公開されるライブラリも将来的に探索できる.

\subsection{実装}
サービスはフロントエンドにReact,バックエンドにFastAPIを用いて実装した.
バックエンドをPythonで実装することで,ライブラリを追加する際事前に構築した機械学習モデルによる推論やサンプル音声の合成を一挙に行うことができる.

ユーザはサービスのホーム画面から各評価軸に対してスライダーを用い1〜7の評価スコアを入力できる.
検索に用いる評価軸はチェックボックスを用いて選択でき,重視しない軸は無視して探索が行われる.
入力された評価スコアとライブラリの評価スコアとの平方ユークリッド距離を計算し,距離が近い順に並び替え探索結果として表示する.

探索結果からは,各ライブラリの名前や評価スコア,アイコン,サンプル音声などを確認できる.
サンプル音声には「かえるのうた」の一節をScoreDraftを用いて事前に合成したものを使用する
.
また各ライブラリの詳細画面を開けば配布URLなどの情報を確認できる.

ユーザはライブラリ追加機能を用いて未登録のライブラリを検索対象として登録できる.
追加する際はライブラリをzipファイルとしてアップロードすると,サービスが自動的に音響特徴量を抽出し構築済みの機械学習モデルで評価スコアを推定する.
アップロード時には合わせて配布ページへのURLやライブラリの名前などのメタデータの入力を要求するほか,サンプル音声の合成も行われ,これらの情報もデータベースに登録される.

% \subsection{ライブラリ探索}
% ユーザは求める声の評価スコアを1〜7の間で入力し,その評価スコアに近いUTAU音源ライブラリを探索できる.
% ユーザは先に述べた7項目のうち、任意の数の評価軸に対してスライダーを操作し,評価スコアを入力する.
% 必ずしも全ての評価軸に対して評価スコアを入力する必要はなく,探索の際重視しない評価軸についてはチェックボックスのチェックを外すことで評価軸を無効化し,その軸を無視して探索を行える.
% サービスは入力された評価スコアに近いライブラリを探索し,その結果をユーザに提示する.
% 評価スコアが近いかの評価には,ユーザに選択された評価軸での平方ユークリッド距離によって判断している.
% 登録された全てのライブラリに対して距離の計算を行い,その昇順にライブラリを表示し求める声に近いライブラリを上位に表示する.

% \subsection{ライブラリ詳細の表示}
% ユーザが探索結果として提示されたライブラリを選択すると,そのライブラリの詳細情報を表示する.
% 詳細情報にはライブラリの名前,推定された評価スコア,サンプル音声,ライブラリの公式サイトや配布ページへのURLなどが含まれる.
% サンプル音声としては,童謡の「かえるのうた」をPythonライブラリ「ScoreDraft」を用い事前に生成したファイルを用いる.
% 本サービスはユーザが求める声に近いライブラリの探索のみを目的としているため,直接ライブラリをダウンロードする機能は提供せず,ユーザが自身でURL先からダウンロードするよう促す.
% また,選択したライブラリの評価スコアを検索欄に転写できる.
% この機能を用いれば,あるライブラリに近い声質のライブラリを探したり,探索結果ページを送り反対に遠い声質のライブラリを探せる.

% \subsection{探索対象ライブラリの追加}
% ユーザはUTAU音源ライブラリをアップロードし,サイト上に存在しないUTAUライブラリを自由に追加できる.
% すでに構築されたモデルを用いて評価スコアやサンプル音声などを自動で生成し,データベースに追加する.
% ユーザがライブラリを追加できるため,より多くのライブラリや,この先新しく生まれ配布されるライブラリに対してもサービスに登録できる.

% Local Variables:
% mode: japanese-LaTeX
% TeX-master: "root"
% End:
