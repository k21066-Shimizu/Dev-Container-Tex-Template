\chapter{はじめに}
\thispagestyle{myheadings}

\section{背景}
近年,人の歌声や喋り声をコンピュータ上で再現する音声合成ソフトウェアは楽曲制作やアナウンスなどに広く利用されており,その普及と同時に音声合成ソフトウェアの数も増加している.
合成音声ソフトウェアの多くは合成音声ライブラリを切り替えて合成される声の種類を変更でき,ユーザが利用できる声の種類はソフトウェアの数以上に存在する.
加えて,いくつかのソフトでは個人のユーザが自分の声からライブラリを作成して第三者に配布でき,そのようなソフトではライブラリ数は非常に多くなる.
例えば,喋り声を対象とした合成音声ソフトCOEIROINKではユーザの作成した音声合成モデルが350キャラクタ分以上配布されているほか\cite{mycoeiroink}ç,歌声を対象とした合成音声ソフトUTAUでは同ソフト上で使用できるUTAU音源ライブラリが7000キャラクタ分以上存在する\cite{vdbutau}.
このように,合成音声ソフトの利用者は使える声に対し非常に多くの選択肢を持っている.

合成音声を利用するシーンにおいて,声の持つイメージや印象は声を選ぶ上で考慮すべき要素である.
例えば,重要な情報をアナウンスする際に舌足らずな声を使うと情報が伝わりにくくなるし,可愛らしいポップな曲調の楽曲に力強い声を使うと聞き手に違和感を与える恐れがある.
用途に合った声質を持つライブラリの採択は,情報を伝える効果や楽曲の表現力の向上に寄与する.
しかし,現状声の持つ印象を知るには実際に聴いてみるのが最も有力な手段であり,数多あるライブラリの生み出す声を十分な数聴き比べ適切な声を採択するには多大な手間と時間を要する.
その結果として,ユーザがライブラリを選ぶ際にその多くが普段の生活の中で聞いた経験のある声や,知っているキャラクターの声を選択していると考えられ,ユーザ全体の中で使われる声には大きな偏りが生じている.
万に近い数存在するライブラリのうち実際にユーザに用いられる声は一握りであり,ほとんどのライブラリはユーザに用いられず埋もれてしまう.

\section{研究目的とアプローチ}
本研究の目的は,多くの合成音声ライブラリの中から,ユーザの用途に合った声・ユーザのイメージする声に近い声を探索する効率的な手法の,実際にユーザが使用できるかたちでの提案である.
このような手法が実現できれば,ユーザは声質に対する自身の要求を入力するだけで,膨大な数の合成音声ライブラリの中から要求に合った声を持つライブラリを効率的に採択できるようになる.
これにより,ユーザは従来のように実際に多くの声を聴き比べる必要がなくなり,声の選択にかかる労力を大幅に削減できる.
また,純粋にライブラリの持つ声質のみでの探索はこれまで埋もれていた多様な声質を持つライブラリの発掘にもつながり,合成音声ライブラリの制作者とユーザの双方への利益が期待できる.

本研究では,ライブラリごとの声に対する印象を数値化する機械学習モデルを構築し,それを用いてユーザの求める声に近いライブラリを探索するサービス「声色見本帳」を提案する.
ユーザは求める声のイメージを評価スコアとして数値化し入力すると,そのスコアに近いライブラリを探索できる.
本研究では特に,数が多く声質に関する既存のアンケートデータが存在するUTAU音源ライブラリを探索対象として選定した.
提案するサービスでは,声に対する印象を複数の印象軸ごとに評価スコアとして数値化する.
この軸は既存のアンケートデータの項目から設定されており,例えば「性別感」という軸では,スコアの高低を女性らしい声・男性らしい声に対応させるなど,ユーザが直感的に理解しやすい評価軸になっている.
各音源に対する評価スコアは,アンケート調査で得られたデータを基に,音源ファイルから各評価スコアを推定する機械学習モデルを作成して付与する.
機械学習モデルと音源ファイルを用いた評価スコアの推定によって,アンケート調査を行なっていないライブラリに対しても機械的に評価スコアを付与できる.
サービスは多くの人が手軽に利用できるようWebアプリケーションとして実装し,実際に使いやすいインターフェースを備える.
さらに探索対象のライブラリをユーザの手で新規に追加できる仕組みを備え,現在では未登録であったり今後作成され新規に公開されるライブラリも将来的に探索できる,網羅的な声質探索システムの実現を目指す.

\section{本論文の構成}
本論文の構成は以下の通りである.
第2章では,関連研究について述べる.
第3章では,ライブラリから評価スコアを推定する機械学習モデルの構築と,構築したモデルの推論精度に対する評価実験について述べる.
第4章では,提案するサービス「声色見本帳」の設計と実装について述べる.
最後に第5章として本論文のまとめと今後の課題について述べる.

% Local Variables:
% mode: japanese-LaTeX
% TeX-master: "root"
% End:
