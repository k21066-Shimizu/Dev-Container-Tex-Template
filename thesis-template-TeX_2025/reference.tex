\thispagestyle{myheadings}
\addcontentsline{toc}{chapter}{\protect\numberline {} 参考文献}

\begin{thebibliography}{参考文献}

\bibitem{mycoeiroink}
COEIROINC,
MYCOEIROINK.
\url{https://coeiroink.com/mycoeiroink/list}

\bibitem{vdbutau}
Vocaloid Database.
\url{https://vocadb.net/Search?searchType=Artist&artistType=UTAU}

\bibitem{pop}
金礪 愛,中野 倫靖,後藤 真孝,菊池 英明,
ポピュラー音楽における歌声の印象評価語を自動推定するシステム.
情報処理学会論文誌,Vol.2013-MUS-100,No.19,pp.1-8,2013

\bibitem{voice_color}
金礪 愛,菊池 英明,
歌唱音声における声質の特徴と想起される色の関係.
日本感性工学会論文誌,17巻,1号,pp.109-118,2018

\bibitem{impression}
金礪 愛,中野 倫靖,後藤 真孝,菊池 英明,
歌声の印象評価尺度の構築に基づく多様な印象の自動推定手法.
情報処理学会論文誌,Vol.57,No.5,pp.1375-1388,2016

\bibitem{ama}
金礪愛,
アマチュア歌唱者に向けた歌声可視化方法の検討.
博士論文,早稲田大学,2018

\bibitem{ong}
山根壮一ほか,
歌声合成システムの音源データに対する声質推定と声質制御.
情報処理学会研究報告,Vol.2015-MUS-108,No.6,pp.1-6,2015

\bibitem{dnn}
横森文哉,大柴まりや,森勢将雅,小澤賢司,
スペクトル包絡情報を入力としたDeep Neural Networkに基づく歌声のための声質評価
情報処理学会音楽情報科学研究会,Vol.2015-MUS-107,No.61,1-6,2015

\bibitem{zcr}
佐賀 圭真,井村 誠孝,
発話音声の聞き取りやすさ向上のための音声特徴量解析.
エンタテインメントコンピューティングシンポジウム2019論文集,2019,pp.84-86,2019

\bibitem{formant}
粕谷 英樹,鈴木 久喜,城戸 健一,
年令, 性別による日本語5母音のピッチ周波数とホルマント周波数の変化,
日本音響学会誌,24巻,6号,pp.355-364,1968

\bibitem{kiite}
産業総合研究所,
音楽印象分析・音楽推薦を駆使して楽曲と出会える音楽発掘サービス「Kiite」を公開
\url{https://www.aist.go.jp/aist_j/press_release/pr2019/pr20190830/pr20190830.html}

\bibitem{tatsu3shiki}
巽式 連続音の録音リスト配布 - 巽のブログ,
\url{https://tatsu3.hateblo.jp/entry/ar426004}

\bibitem{utausurvey}
ニコニコ大百科,
UTAU音源声質アンケートとは,
% \url{https://dic.nicovideo.jp/a/utau%E9%9F%B3%E6%BA%90%E5%A3%B0%E8%B3%AA%E3%82%A2%E3%83%B3%E3%82%B1%E3%83%BC%E3%83%88}
\url{https://dic.nicovideo.jp/a/utau音源声質アンケート}

\end{thebibliography}
% Local Variables:
% mode: japanese-LaTeX
% TeX-master: "root"
% End:
