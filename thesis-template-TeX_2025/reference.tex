\thispagestyle{myheadings}
\addcontentsline{toc}{chapter}{\protect\numberline {} 参考文献}

\begin{thebibliography}{参考文献}

\bibitem{mycoeiroink}
COEIROINC,
MYCOEIROINK.
\url{https://coeiroink.com/mycoeiroink/list}

\bibitem{vdbutau}
Vocaloid Database.
\url{https://vocadb.net/Search?searchType=Artist&artistType=UTAU}

% \bibitem{pop}
% 金礪 愛,中野 倫靖,後藤 真孝,菊池 英明,
% ポピュラー音楽における歌声の印象評価語を自動推定するシステム.
% 情報処理学会論文誌,Vol.2013-MUS-100,No.19,pp.1-8,2013

% \bibitem{voice_color}
% 金礪 愛,菊池 英明,
% 歌唱音声における声質の特徴と想起される色の関係.
% 日本感性工学会論文誌,17巻,1号,pp.109-118,2018

% \bibitem{ama}
% 金礪愛,
% アマチュア歌唱者に向けた歌声可視化方法の検討.
% 博士論文,早稲田大学,2018

\bibitem{impression}
金礪 愛,中野 倫靖,後藤 真孝,菊池 英明,
歌声の印象評価尺度の構築に基づく多様な印象の自動推定手法.
情報処理学会論文誌,Vol.57,No.5,pp.1375-1388,2016

\bibitem{rnd_forest}
田中 晶之,中村 嘉志,
ランダムフォレスト法を用いた歌声が聞き手に与える印象の単語推定に関する研究.
国士舘大学理工学部紀要,vol.15,No.15,pp.23-28,2022

\bibitem{pop_seiku}
平山 健太郎,伊藤 克亘,
ポピュラー歌唱における高音域の声区と発声状態の判別手法.
情報処理学会研究報告,Vol.2012,No.16,pp.1-6,2012

\bibitem{vibrato_monomane}
鈴木 千文,坂野 秀樹,旭 健作,森勢 将雅,
歌唱音声の類似度評価を目的とした基本周波数の変動量を反映するビブラート特徴量の提案.
電気学会論文誌C(電子・情報・システム部門誌),Vol.137,No.12,pp.1607-1614,2017

\bibitem{monomane_tec}
山本 雄也,中野 倫靖,後藤 真孝,寺澤 洋子,
ポピュラー音楽の模倣歌唱における歌唱テクニック分析と楽譜情報との対応付け.
情報処理学会論文誌,Vol.64,No.10,pp.1423-1437,2023

\bibitem{ong},
山根 壮一,小林 和弘,戸田 智基,中野 倫靖,後藤 真孝,ニュービッグ グラム,サクリアニ サクティ,中村 哲,
歌声合成システムの音源データ検索のための声質評価値推定法.
情報処理学会研究報告,Vol.2015-MUS-108,No.6,pp.1-6,2015

\bibitem{dnn}
横森 文哉,大柴 まりや,森勢 将雅,小澤 賢司,
スペクトル包絡情報を入力としたDeep Neural Networkに基づく歌声のための声質評価.
情報処理学会音楽情報科学研究会,Vol.2015-MUS-107,No.61,pp.1-6,2015

\bibitem{talk_search}
佐治 拓樹,小林 和弘,石黒 祥生,戸田 智基,大谷 健登,西野 隆典,武田 一哉,
声質の可視化を用いた所望音声検索システムの提案.
情報処理学会研究報告,Vol.2022-MUS-133,No.6,1-5,2022

\bibitem{kiite}
産業総合研究所,
音楽印象分析・音楽推薦を駆使して楽曲と出会える音楽発掘サービス「Kiite」を公開.
\url{https://www.aist.go.jp/aist_j/press_release/pr2019/pr20190830/pr20190830.html}
(2025年01月23日閲覧)

\bibitem{mfcc_cute}
小島 俊,齋藤 毅,三好 正人,
歌声における印象評価と音響特徴量の関係について.
電子情報通信学会技術研究技術報告,vol.111,No.471,pp.49-53,2021

\bibitem{zcr}
佐賀 圭真,井村 誠孝,
発話音声の聞き取りやすさ向上のための音声特徴量解析.
エンタテインメントコンピューティングシンポジウム2019論文集,2019,pp.84-86,2019

\bibitem{formant}
粕谷 英樹,鈴木 久喜,城戸 健一,
年令, 性別による日本語5母音のピッチ周波数とホルマント周波数の変化.
日本音響学会誌,24巻,6号,pp.355-364,1968

\bibitem{tatsu3shiki}
巽式 連続音の録音リスト配布 - 巽のブログ.
\url{https://tatsu3.hateblo.jp/entry/ar426004}
(2025年01月23日閲覧)

\bibitem{utausurvey}
ニコニコ大百科,
UTAU音源声質アンケートとは.
% \url{https://dic.nicovideo.jp/a/utau%E9%9F%B3%E6%BA%90%E5%A3%B0%E8%B3%AA%E3%82%A2%E3%83%B3%E3%82%B1%E3%83%BC%E3%83%88}
\url{https://dic.nicovideo.jp/a/utau音源声質アンケート}
(2025年01月23日閲覧)

\end{thebibliography}
% Local Variables:
% mode: japanese-LaTeX
% TeX-master: "root"
% End:
