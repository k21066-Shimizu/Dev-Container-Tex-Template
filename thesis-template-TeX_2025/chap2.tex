\thispagestyle{myheadings}
\chapter{関連研究}
\label{sec:format}

\section{歌声に対して印象を推定する研究}
人の歌声を対象に,その印象を推定する研究は数多く存在する.
金礪らはアマチュア女性歌唱者の歌声を対象に,同じメロディを歌った歌声から音響特徴量を抽出し,「迫力性」「丁寧さ」「明るさ」の3つの評価語に対するスコアを推定するモデルを作成している\cite{impression}.
この研究では,歌声から1msecごとに音響特徴量を抽出し,評価語との関係を重回帰分析によってモデル化し,そのモデルを用いて評価語のスコアを推定している.
また田中らの研究\cite{rnd_forest}では,既存の楽曲から抽出した歌声を用い歌声の印象評価語を推定するモデルを作成している.
この研究では歌声に対する印象評価語を楽曲ごとに付与した後,抽出した歌声を歌詞のフレーズごとに分割し音響特徴量を抽出,ランダムフォレスト法で学習した分類器を作成している.
結果として,「透明感のある」という印象評価語に対して高い精度での推定を実現している.
これらの研究から,歌声に対して印象を推定するための評価軸を用いた評価スコア付与が可能である点や,それを用いた歌声の表現は一つの手段として妥当なものであると考えられる.

これらの研究は実際に人が歌唱している音声から音響特徴量を抽出しているが,実際の歌唱には歌唱者の声質の差異以外にも,例えばビブラートの強さやしゃくりの癖,声区の使い方など歌い方の特徴が表れる.
ビブラートやしゃくりはピッチの変動によって生じるテクニックであり,F0の変動量に大きな影響を与える.
他にも音響特徴量から声区の使い方を推定する研究\cite{pop_seiku}があるように,歌い方は音響特徴量に影響を与えており,実際の歌唱から抽出した音響特徴量には歌い方の特徴が含まれていると言える.
歌い方の特徴は歌声の個人性知覚に対して大きく影響を与える\cite{vibrato_monomane}\cite{monomane_tec}ため,音響特徴量に含まれる歌い方の特徴は,歌声の印象推定にも影響を与えていると予想できる.
一方で,本研究で対象とするUTAUを含め多くの合成音声において歌い方はライブラリではなくソフトの使用者に依存するため,それらの情報にまつわるライブラリごとの音響特徴量取得が不可能である.
したがって,本研究では音響特徴量に含まれているビブラートやしゃくりなどの歌い方の特徴量に頼らない分析が必要となる.

\section{合成音声の歌声にスコアを付与する研究}
本研究でも扱うUTAU音源ライブラリとUTAU音源声質アンケート,あるいはその規格を用いた研究が存在する.
山根らによる研究\cite{ong}では,40個のUTAU音源ライブラリを用いて合成した音声データから声質特徴量ベクトルを抽出し,ベクトルからスコアを推定するモデルを重回帰分析とカーネル回帰分析を用いて作成している.
スコアの推定に用いる音声データとして短音節発声,話声,歌声の3種類を用いた際の推定精度を比較しており,その結果として音声データの種類による推定精度の差異は少ないとされている.
横森らによる研究\cite{dnn}では80個のUTAU音源ライブラリから特徴量としてMFCCを抽出し,Deep Neural Network(DNN)を用いて声質の評価スコアを推定し,評価実験を行なっている.
これらの研究ではスコアを推定するモデルの作成に止まっており,そのスコアを用いたライブラリの探索システムやサービスの作成は行われていない.

\section{音声を可視化し探索する研究・サービス}
音声をある形式で可視化し,それを探索に用いる研究やサービスが存在する.
佐治らは,喋り声に対して話者特徴量を抽出して可視化し,その視覚情報をもとに音声を取り出す検索システムを作成している\cite{talk_search}.
32次元の値として得た話者特徴量を主成分分析によって2次元に圧縮し,2次元空間上にプロットし話者特徴量の分布を可視化している.
話者特徴量は人間には直感的に理解しにくく,2次元空間上にプロットしてもその配置に一貫した意味があるかは不明である.
しかしある程度似た声質を持つ話者は近い位置にプロットされる傾向があり,結果として話者の性別ごとにある程度固まってプロットされるため,これを用いた話者の探索が可能だった.
また,同研究では2次元空間上へのプロットと声真似による入力を組み合わせた検索手法を提案している.
これはユーザに地声と求める声の声真似を入力させ,それらの特徴量の差を「ユーザがイメージする声質へ向かうベクトル」と仮定し2次元空間上の範囲を限定する手法である.
評価実験によってこの手法は探索時間の短縮が可能だと示されている.
この研究は人にとって直感的に理解できない特徴量を用いているものの,その特徴量を2次元平面として可視化し,さらにユーザの声を入力に用いて探索を効率化している.

クリプトン・フューチャー・メディア株式会社と産業総合研究所によって開発された音楽発掘サービス「Kiite(キイテ)」では、その機能の一つとして「Kiite Radar」が提供されている\cite{kiite}.
このサービスでは,楽曲を知名度やニコニコ動画上でのマイリスト率,楽曲を歌っているキャラクターなどの一般的な楽曲情報での絞り込みに限らず,楽曲の解析によって得られた曲の声質や踊りたくなるかどうかの印象をスライダーで設定し,その条件に合致する楽曲を探索できる.
本研究で扱っている声質を楽曲の絞り込みに用いられるが,「かっこいい」と「かわいい」を両端に持つスライダー1つのみが用意されており,ユーザはこのスライダーを動かして楽曲を探索できる.
加えて,全ての楽曲を分析した印象に基づいて2次元空間上にプロットされた印象マップが提供されている.
激しい曲は左上に,軽快な曲は右上にプロットされるなど楽曲の配置には一定の傾向があり,ユーザはこれらを用いて直感的に楽曲を探索できる.
対象が歌声でなく楽曲である点は異なるが,本来数値ではない印象を数値化し,それを用いて可視化・探索するという点で本研究と共通する部分があり,サービスを提供する上で参考にできる.

% Local Variables:
% mode: japanese-LaTeX
% TeX-master: "root"
% End:
