\thispagestyle{myheadings}
\chapter{関連研究}
\label{sec:format}

\section{歌声に対して印象を推定する研究}
人の歌声を対象に,その印象を推定する研究は数多く存在する.
金礪らは歌唱の情報を可視化する手法を提案している\cite{impression}.
この研究では歌声から音響特徴量を抽出し,「迫力性」「丁寧さ」「明るさ」の3つの評価語に対するスコアを推定するモデルを作成している.
同著者による別の研究では,歌声の声質に対してイメージされる色を推定する試みも行っている\cite{voice_color}.

% 本研究に関連する研究として,人間の歌声の情報を理解しやすい形で可視化する研究\cite{impression, ama}がある.
% この研究では,ある程度の長さがある歌声と,そのうちの瞬間的な長さの歌声を用い,それぞれで印象を推定するモデルを作成している.
% ある程度の長さがある歌声では,迫力性,丁寧さ,明るさの3軸に対してそれぞれのスコアを推定するモデルを作成し,推定を行なった.
% 結果として,人の間でも印象の評価が大きく揺れるような歌唱などの例外を除き,十分な精度で印象を推定できた.
% また歌声のうち瞬間的な音声からは,印象に加えて声に対してイメージされる色を推定する試みも行っている.
% 彩度など色の要素と表現語の一つとして挙げられた活動性との関係が見られるなど,声質の色での表現に対して一定の有効性が示されたものの,他の要素との関係については今後の課題とされている.
% この研究から,声質へのなんらかの評価軸を用いた評価スコア付与が可能である点や,それを用いた声質の表現は一つの手段として妥当なものであると考えられる.

\section{合成音声の歌声にスコアを付与する研究}
UTAU音源を対象に声質に対し評価スコアを付与し,そのスコアを用いて音源を探索するシステムを提案する研究\cite{ong}が存在する.
この研究でも本研究と同じくUTAUに評価スコアを付与,探索システムを構築し,実際にユーザが求める声を探索できるかを確認している.
スコアの推定にはUTAUを用いて合成された音声データを用い,重回帰分析とカーネル回帰分析での推定制度の比較を行なっている.
また,推定されたスコアを用いて音源を探索する際には,ある2つのスコア行列間のユークリッド距離を目標類似距離とし,その逆数として定義した目標類似度を用いて音源を探索している.
評価実験では,ユーザのイメージする声に近いスコアを入力し,目標類似度の高いライブラリを提示した.するとユーザが求めるような声を持つライブラリを探索できる結果が示された.
一方でこの研究では,探索システムの実装に留まっており,探索アルゴリズムの検討や実際にユーザが利用できるサービスの提案は行われていない.

\section{可視化・探索するシステムを提供するサービス}
クリプトン・フューチャー・メディア株式会社と産業総合研究所によって開発された音楽発掘サービス「Kiite(キイテ)」では、その機能の一つとして「Kiite Radar」が提供されている\cite{kiite}.
ユーザは,楽曲の知名度やニコニコ動画上でのマイリスト率,楽曲を歌っているキャラクターなどの一般的な楽曲情報での絞り込みに限らず,楽曲の解析によって得られた曲の声質や踊りたくなるかどうかの印象をスライダーで設定し,その条件に合致する楽曲を探索するできる.
加えて,全ての楽曲を分析した印象に基づいて2次元平面上にプロットされた印象マップが提供されている.
激しい曲は左上に,軽快な曲は右上にプロットされるなど楽曲の配置には一定の傾向があり,ユーザはこれらを用いて直感的に楽曲を探索できる.
対象が歌声でなく楽曲である点は異なるが,本来数値ではない印象を数値化し,それを用いて可視化・探索するという点で本研究と共通する部分があり,サービスを提供する上で参考にできる.

% Local Variables:
% mode: japanese-LaTeX
% TeX-master: "root"
% End:
